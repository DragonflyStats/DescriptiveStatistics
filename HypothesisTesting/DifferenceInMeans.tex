	\documentclass[a4paper,12pt]{article}
%%%%%%%%%%%%%%%%%%%%%%%%%%%%%%%%%%%%%%%%%%%%%%%%%%%%%%%%%%%%%%%%%%%%%%%%%%%%%%%%%%%%%%%%%%%%%%%%%%%%%%%%%%%%%%%%%%%%%%%%%%%%%%%%%%%%%%%%%%%%%%%%%%%%%%%%%%%%%%%%%%%%%%%%%%%%%%%%%%%%%%%%%%%%%%%%%%%%%%%%%%%%%%%%%%%%%%%%%%%%%%%%%%%%%%%%%%%%%%%%%%%%%%%%%%%%
\usepackage{eurosym}
\usepackage{vmargin}
\usepackage{amsmath}
\usepackage{graphics}
\usepackage{framed}
\usepackage{epsfig}
\usepackage{subfigure}
\usepackage{enumerate}
\usepackage{fancyhdr}

\setcounter{MaxMatrixCols}{10}
%TCIDATA{OutputFilter=LATEX.DLL}
%TCIDATA{Version=5.00.0.2570}
%TCIDATA{<META NAME="SaveForMode"CONTENT="1">}
%TCIDATA{LastRevised=Wednesday, February 23, 201113:24:34}
%TCIDATA{<META NAME="GraphicsSave" CONTENT="32">}
%TCIDATA{Language=American English}

\pagestyle{fancy}
\setmarginsrb{20mm}{0mm}{20mm}{25mm}{12mm}{11mm}{0mm}{11mm}
\lhead{MS4222} \rhead{Kevin O'Brien} \chead{Hypothesis Testing} %\input{tcilatex}

\begin{document}
\section*{Worked Example: Comparing Recovery Times}
\begin{itemize}
\item Two sets of patients are given courses of treatment under two different drugs. \item 
The benefits
derived from each drug can be stated numerically in terms of the recovery times measured in days
\item We wish to see if there is a statistically significant difference betwee the two groups.
\item The sample size, mean and standard deviations for both groups, denominated in recovery days, are given below.
\end{itemize}
\begin{center}
 \begin{tabular}{|c|c|c|c|}
\hline Group & Sample Size & Mean & Std. Dev.  \\ 
\hline 1 & $n_1$ = 40 & $\bar{x}_1$ = 3.3 days  &  $s_1 = 1.524$ \\ 
\hline 2 & $n_2$ = 45 & $\bar{x}_2$ = 4.3 days & $s_2 = 1.951 $ \\ 
\hline 
\end{tabular} 

   
\end{center}


\noindent \textbf{Hypothesis Testing Procedure}
\begin{itemize}
\item
The first step in hypothesis testing is to specify the null hypothesis and an alternative hypothesis.
\item When testing differences between mean recovery times, the null hypothesis is that the two population means are equal.
\item That is, the null hypothesis is:\\
$H_0: \mu_1 = \mu_2$ (The population means are equal)\\
$H_1: \mu_1 \neq \mu_2$ (The population means are different)\\
\end{itemize}


\noindent \textbf{Computing the Test Statistic}
\begin{itemize}
\item \textbf{Point Estimate}: The observed difference in means is $\bar{x}_1-\bar{x}_2 \;= \;4.3-3.3 \;$= 1 day.
\vspace{0.2cm}
\item The relevant formula for the standard error is
\[ S.E(\bar{x}_1 - \bar{x}_2) = \sqrt{{s^2_1\over n_1}+{s^2_2 \over n_2}} \]
\[ S.E(\bar{x}_1 - \bar{x}_2) = \sqrt{{(1.524)^2 \over 40}+{(1.951)^2 \over 45}}   \]\vspace{0.2cm}
\[ S.E(\bar{x}_1 - \bar{x}_2) = 0.377\mbox{ days}\]
\end{itemize}




\begin{itemize}
\item The Test statistic is therefore
\[ TS = {\mbox{Point Estimate} - \mbox{Expected Difference under } H_0 \over \mbox{Std. Error}} \]

\[TS = {1 - 0 \over 0.377 } = 2.65 \]


\end{itemize}


\noindent \textbf{The Critical Value}
\begin{itemize}

\item This is a Two-Tailed Test, therefore $k = 2$.
\item The significance level $\alpha = 0.05$. 
\item Also we have two large samples ($n_1 + n_2 > 30$). 
\item The Critical Value is therefore 1.96.
\end{itemize}

\noindent \textbf{Decision Rule}
\[ |TS| > CV ?  \]
\begin{itemize}
\item If Yes: Reject the null Hypothesis
\item If No : Fail to reject the Null Hypothesis
\end{itemize}
$|2.65|$ is greater than 1.96. We reject the null hypothesis.  There is enough evidence to suggest that there is a difference in drug treatments.There is a significant difference in the effectiveness of the two drug treatments.


\section*{Using a $p-$value approach with \texttt{R}}
\begin{itemize}
% \item The Test statistic is therefore
% \[ TS = {\mbox{observed} - \mbox{null} \over \mbox{Std. Error}}  = {1 - 0 \over 0.377 } = 2.65 \]
\item Lets compute the p-value of the Test Statistic : \\
p-value = $P(z \geq 2.65) = 0.0040$
\begin{verbatim}
> 1-pnorm(2.65)
[1] 0.004024589
\end{verbatim}
\item (Remark: Two Tailed Test, therefore $k = 2$, and $\alpha = 0.05$)
\item What is this value smaller than threshold $\alpha / k$? \\
\item $\alpha / k$ = $0.05/2$ = 0.0250? Yes the p-value is smaller than this.
\item \textbf{Conclusion:} we reject the null hypothesis. There is a significant different between both drugs, in terms of recovery times.

\end{itemize}





\end{document}

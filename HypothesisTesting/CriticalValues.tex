	\documentclass[a4paper,12pt]{article}
%%%%%%%%%%%%%%%%%%%%%%%%%%%%%%%%%%%%%%%%%%%%%%%%%%%%%%%%%%%%%%%%%%%%%%%%%%%%%%%%%%%%%%%%%%%%%%%%%%%%%%%%%%%%%%%%%%%%%%%%%%%%%%%%%%%%%%%%%%%%%%%%%%%%%%%%%%%%%%%%%%%%%%%%%%%%%%%%%%%%%%%%%%%%%%%%%%%%%%%%%%%%%%%%%%%%%%%%%%%%%%%%%%%%%%%%%%%%%%%%%%%%%%%%%%%%
\usepackage{eurosym}
\usepackage{vmargin}
\usepackage{framed}
\usepackage{amsmath}
\usepackage{graphics}
\usepackage{epsfig}
\usepackage{subfigure}
\usepackage{enumerate}
\usepackage{fancyhdr}

\setcounter{MaxMatrixCols}{10}
%TCIDATA{OutputFilter=LATEX.DLL}
%TCIDATA{Version=5.00.0.2570}
%TCIDATA{<META NAME="SaveForMode"CONTENT="1">}
%TCIDATA{LastRevised=Wednesday, February 23, 201113:24:34}
%TCIDATA{<META NAME="GraphicsSave" CONTENT="32">}
%TCIDATA{Language=American English}

\pagestyle{fancy}
\setmarginsrb{20mm}{0mm}{20mm}{25mm}{12mm}{11mm}{0mm}{11mm}
\lhead{MS4222} \rhead{Kevin O'Brien} \chead{Hypothesis Testing} %\input{tcilatex}

\begin{document}


%%%%%%%%%%%%%%%%%%%%%%%%%%%%%%%%%%%%%%%%%%%%%%%%%%%%%%%%%%%%%%%%%%
\section*{The Critical Value}

\begin{itemize}
	\item The critical value(s) for a hypothesis test is a threshold to which the value of the test statistic in sample is compared to determine whether or not the null hypothesis is rejected.
		\item We will use the initials CV for the sake of brevity.
	\item The critical value for any hypothesis test depends on the significance level at which the test is carried out, and whether the test is one-sided or two-sided.

	\item The critical value is determined the exact same way as quantiles for confidence intervals; using Murdoch Barnes table 7.
\end{itemize}


%%%%%%%%%%%%%%%%%%%%%%%%%%%%%%%%%%%%%%%%%%%%%%%%%%%%%%%%%%%%%%%%%%
\subsection*{Determining the Critical value}
\begin{itemize}
	\item A pre-determined level of significance $\alpha$ must be specified. Usually it is set at 5\% (0.05).
	\item The number of tails must be specified, either one tailed or two tailed, i.e. $k$ is either 1 or 2.
	\item Sample size is an issue. We must decide whether to use $n-1$ degrees of freedom or $\infty$ degrees of freedom, depending on the sample size in question.
	\item The manner by which we compute critical value is identical to the way we compute quantiles. 
	% We will consider this in more detail during tutorials.
	% \item For the time being we will use 1.96 as a critical value.
\end{itemize}

%=============================================================%

%	\subsection*{Critical value}
%	A critical value is any value that separates the critical region ( where we reject the null hypothesis) for that tha values of the test statistic that do not lead to a rejection of the null hypothesis.


%%%%%%%%%%%%%%%%%%%%%%%%%%%%%%%%%%%%%%%%%%%%%%%%%%%%%%%%%

\subsection*{Critical Region }
	
	Remark: The absolute value function of some value x is denoted $|x|$. It is the magnitude of the value without consideration of whether the value is positive or negative.
	
	
	\begin{itemize}
		\item Let TS denote Test Statistic and CV denoted Critical Value.
		\item $|TS| > CV$ Then we reject null hypothesis.
		\item $|TS| \leq CV$ Then we \textbf{fail to reject} null hypothesis.
		\item Suppose TS = 2.99, CV = 1.96
%		\item For our die-throw example; TS = 2.99, CV = 1.96.
		\item Here $|2.99| > 1.96$ we reject the null hypothesis that the die is fair.
%		\item Consider this in the context of ``proof". 
		%(More on this in next class)
	\end{itemize}

% \subsection*{Critical Value}
%
%
% The critical region ( or rejection region ) is the set of all values of % the test statistic that causes us to reject the null hypothesis.

%===================================================================%





\end{document}

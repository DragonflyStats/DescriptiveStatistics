\documentclass[a4paper,12pt]{article}
%%%%%%%%%%%%%%%%%%%%%%%%%%%%%%%%%%%%%%%%%%%%%%%%%%%%%%%%%%%%%%%%%%%%%%%%%%%%%%%%%%%%%%%%%%%%%%%%%%%%%%%%%%%%%%%%%%%%%%%%%%%%%%%%%%%%%%%%%%%%%%%%%%%%%%%%%%%%%%%%%%%%%%%%%%%%%%%%%%%%%%%%%%%%%%%%%%%%%%%%%%%%%%%%%%%%%%%%%%%%%%%%%%%%%%%%%%%%%%%%%%%%%%%%%%%%
\usepackage{eurosym}
\usepackage{vmargin}
\usepackage{amsmath}
\usepackage{graphics}
\usepackage{epsfig}
\usepackage{subfigure}
\usepackage{framed}
\usepackage{multicol}
\usepackage{enumerate}
\usepackage{fancyhdr}

\setcounter{MaxMatrixCols}{10}
%TCIDATA{OutputFilter=LATEX.DLL}
%TCIDATA{Version=5.00.0.2570}
%TCIDATA{<META NAME="SaveForMode"CONTENT="1">}
%TCIDATA{LastRevised=Wednesday, February 23, 201113:24:34}
%TCIDATA{<META NAME="GraphicsSave" CONTENT="32">}
%TCIDATA{Language=American English}

\pagestyle{fancy}
\setmarginsrb{20mm}{0mm}{20mm}{25mm}{12mm}{11mm}{0mm}{11mm}
\lhead{MS4222} \rhead{Kevin O'Brien} \chead{Week 4 Tutorial} %\input{tcilatex}

\begin{document}
\subsection*{Outcomes}
\begin{itemize}
    \item Descriptive Statistics
    \item Boxplots
    \item Binomial Distribution
\end{itemize}
\subsection*{Questions}	
\begin{enumerate}	
\item Compute the range of the following Data Set.

\[\{65,73,89,56,73,52,47\}\]


\item Computing the Sample Mean 
	
	
	\begin{enumerate}[(a)]
		\item Compute the sample mean for the following data set:\[x = \{ 14, 22, 19, 25, 20, 18, 17, 29, 27 \}\]
		
		
		% Mean = 21 
		% n = 9
		% sum = 189
		
		%-----------------------------------%
		
		\item Compute the sample mean for the following data set:
		\[x =\{10, 3, -7, -9, 7, 2, 4, 19, 25, -6\}\]
		
		% sum = 48
		% n = 10
		% mean = 4.8
		
		%-------------------------------------%
		
		
		\item Compute the sample mean for the following data set:
		\[ x =	\{6, 4, 5, 0, -8, 17, -7, 23, 0, 9\}\]
		
		% mean = 4.9
		% n=10
		
	\end{enumerate}

\item 	For each of the following data sets, sort the values into ascending order and state the sample size.
	Then compute the median and interquartile range.
	
	
	\begin{itemize}
		\item[(i)] $\{31, 40, 33, 39, 11, 12, 28, 23, 18, 15, 34, 24\}$ \smallskip
		
		
		% W <- c(31L, 40L, 33L, 39L, 11L, 12L, 28L, 23L, 18L, 15L, 34L, 24L)
		% Answers (n=12)
		% -  Median : 26
		% -  Q1     : 16.5
		% -  Q3     : 33.5
		
		\item[(ii)]	$\{31, 21, 37, 27, 28, 11, 10, 23, 35, 34, 22\}$ \smallskip
		
		
		% X <- c(31L, 21L, 37L, 27L, 28L, 11L, 10L, 23L, 35L, 34L, 22L)
		% 
		% Answers (n=11)
		% -  Median : 37
		% -  Q1     : 21
		% -  Q3     : 34
		
		
		
		\item[(iii)] 	$\{38, 25, 40, 32, 30, 16, 11, 36, 31, 12\}$ \smallskip
		
		% Y <- c(38L, 25L, 40L, 32L, 30L, 16L, 11L, 36L, 31L, 12L)
		% 
		% Answers (n=10)
		% -  Median : 30.5
		% -  Q1     : 16
		% -  Q3     : 36
		
		\item[(iv)] 	$\{28, 13, 33, 23, 25, 26, 40, 16, 38\}$ \smallskip
		
		
		% Z <- c(28L, 13L, 33L, 23L, 25L, 26L, 40L, 16L, 38L)
		% 
		% Answers (n=9)
		% -  Median : 26
		% -  Q1     : 19.5
		% -  Q3     : 35.5
	\end{itemize}
	
	%-------------------------------------%
	
\item Computing Sample Variance and Standard Deviation
	
	\begin{enumerate}[(a)]
		
		\item Compute the sample variance and standard deviation of the following data set:
		\[x =\{5, -11, 9, 20, 18, 21, -8, 19, 8\}\; \; \; ( \bar{x} = 9) \]
		
		\item Compute the sample variance and standard deviation of the following data set:
		\[x =\{32, 20, 15, 49, 30, 28, 33, 35, 34, 44\}
		\; \; \; ( \bar{x} = 32) \]
		
		
		\item Compute the sample variance and standard deviation of the following data set:
		\[
		x =\{19, 16, 35, 34, 27, 17, 20\} \; \; \; ( \bar{x} = 24) \]
		
		
	\end{enumerate}
\newpage	
	\item The following data give the marks of 10 students in a test (out of 20 marks). 
	
	\[12, 17, 7, 11, 18, 6, 14, 15, 11, 9.\]
	
	Calculate the following descriptive statistics

	\begin{itemize}   
		\item[(a)] the median,    
		\item[(b)] the mean,     
		\item[(c)] the range,    
		\item[(d)] the standard deviation, 
		\item[(e)] The inter-quartile range.
	\end{itemize}

	
\item 
The grades of a mid-term assessment for a class of students is tabulated as follows;
\begin{table}[ht]
\begin{center}
\begin{tabular}{|rrrrrrrrrr|}
\hline
31 & 36 & 37 & 38 & 38 & 38 & 39 & 39 & 39 & 39 \\
41 & 44 & 44 & 45 & 45 & 45 & 46 & 46 & 46 & 47 \\
47 & 47 & 48 & 48 & 51 & 53 & 54 & 55 & 57 & 59 \\
60 & 61 & 61 & 63 & 74 & 75 & 81 & 81 & 82 & 89 \\
\hline
\end{tabular}
\end{center}
\end{table}
\begin{enumerate}[(a)]
% \item Comment on the shape of the histogram. 
%\item Based on the shape of the histogram, what is the best measure of centrality and variability?
\item Construct a box plot for the above data. Clearly demonstrate how all of the necessary values were computed.
\end{enumerate}
\item 
%-----------------------------------------------------------------%
The heights for a group of forty rowing club members are tabulated as follows;
\begin{table}[ht]
\begin{center}
\begin{tabular}{|rrrrrrrrrr|}
\hline
141 & 148 & 149 & 149 & 155 & 156 & 167 & 169 & 169 & 170 \\
171 & 173 & 175 & 176 & 177 & 179 & 182 & 182 & 183 & 183 \\
183 & 184 & 184 & 184 & 185 & 185 & 185 & 186 & 186 & 189 \\
191 & 191 & 191 & 191 & 192 & 192 & 192 & 193 & 194 & 199 \\
\hline
\end{tabular}
\end{center}
\end{table}
\begin{enumerate}[(a)]

\item Construct a box plot for the above data. Clearly demonstrate how all of the necessary values were computed.
\end{enumerate}




\item
I throw a coin 5 times.  Using the Binomial Distribution formula, calculate the probability that
\begin{multicols}{2}
\begin{enumerate}[(a)]
\item  I throw no heads
\item  I throw one head
\item  I throw exactly 3 heads
\item  I throw at least 2 heads
\end{enumerate}
\end{multicols}
\newpage
\item 
A die is thrown 5 times. Using the Binomial Distribution formula, calculate the probability of
\begin{enumerate}[(a)]
\item  Obtaining exactly one six.
\item  Obtaining at least one six.
\item  Obtaining at least one six, given that not more than two sixes are thrown.
\end{enumerate}

\item
The probability that a component produced in a certain factory is defective is 0.02. A batch contains 100
components.
\begin{enumerate}[(a)]
\item  What is the exact distribution of the number of defective components in a batch?
\item  Calculate the probability that none of the components in a batch are defective.
\item  Calculate the probability that there is more than one defective component in a batch.
\end{enumerate}

\item
Suppose X is a binomial random variable with
$X \sim Bin(n, p)$.
\begin{enumerate}[(a)]
\item  Describe, in your own words, what is meant by the expected value of X.
\item  Compute the expected value, the variance and the standard deviation for the following scenarios

\begin{multicols}{2}
\begin{enumerate}[(i)]
	\item $X \sim Bin(n = 10, p = 0.40)$
	\item $X \sim Bin(n = 15, p = 0.25)$
	\item $X \sim Bin(n = 20, p = 0.30)$
	\item $X \sim Bin(n = 50, p = 0.20)$
	\item $X \sim Bin(n = 200, p = 0.10)$
	\item $X \sim Bin(n = 1000, p = 0.01)$
\end{enumerate}
\end{multicols}
\end{enumerate}

\item 
Suppose a gambler is playing a simple coin flip game. 
The gambler does not know that the coin has been tampered with such that the probability of a Head is 47\%.

Suppose the gamble plays this coin flip game nine times. 
What is the probability that he wins precisely 3 times.

\item Components are placed into containers containing 100 items. After an inspection of a large number of containers the average number of defective items was found to be 10 with a standard deviation of three.
Is the binomial distribution a good useful distribution, given the observed data?


	
\item Suppose there are twelve multiple choice questions in an English class quiz. Each question has five possible answers, and only one of them is correct. Find the probability of having exactly four correct answers if a student attempts to answer every question at random.	
	
	\item Suppose we have a biased coin which yields a ``head" only $48\%$ of the time.
\begin{enumerate}[(a)]
	

	\item Is this a binomial experiment?  why?
	\item What is the probability of 4 heads in 7 throws?

\end{enumerate}
\item Suppose there is a container that contains 6 items.  The probability that any one of these items is defective is 0.3. Suppose all six items are inspected. 
	\begin{enumerate}[(a)]
		\item What is the probability of 3 defective components?
		\item What is the probability of 4 defective components?
	\end{enumerate}
	
\item A biased coin yields `Tails' on $48\%$ of throws. Consider an experiment that consists of throwing this coin 11 times.
\begin{enumerate}[(a)]
	\item Evaluate the following term $^{11}C_2$.
	\item Compute the probability of getting two `Tails' in this experiment.
\end{enumerate}



\item 
An inspector of computer parts selects a random sample of components
from a large batch to decide whether or not to audit the full batch.

%---------------------------------------------%
\begin{enumerate}[(a)]
	\item lf 20\% or more of the sample is defective, the entire batch is
	inspected, Calculate the probability of this happening if it is
	thought that the population contains 4\% defective components and
	a sample of 20 is selected.
	\item lf 10\% or more of the sample is defective, the entire batch is
	inspected. Calculate the probability of this happening if it is
	thought that the population contains 4\% defective components and
	a sample of 50 is selected.
	
\end{enumerate}


\item	Flextronics supply PCB boards to Dell.  You are a production manager with Dell.  There is a constant probability of 0.01 that a board will be defective.  You select 20 boards at random.  What is the probability that:
\begin{itemize}
	\item[(a)]	0 boards will be defective
	\item[(b)]	1 or more boards will be defective
	\item[(c)]	2 or less boards will be defective			

\end{itemize}

	\end{enumerate}
\begin{framed}
\noindent Binomial Distribution Formula:

\[\Pr(X = k) = ^nC_k p^k
(1 - p)^{n-k}\]	
\end{framed}
	\end{document}

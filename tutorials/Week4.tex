\documentclass[a4paper,12pt]{article}
%%%%%%%%%%%%%%%%%%%%%%%%%%%%%%%%%%%%%%%%%%%%%%%%%%%%%%%%%%%%%%%%%%%%%%%%%%%%%%%%%%%%%%%%%%%%%%%%%%%%%%%%%%%%%%%%%%%%%%%%%%%%%%%%%%%%%%%%%%%%%%%%%%%%%%%%%%%%%%%%%%%%%%%%%%%%%%%%%%%%%%%%%%%%%%%%%%%%%%%%%%%%%%%%%%%%%%%%%%%%%%%%%%%%%%%%%%%%%%%%%%%%%%%%%%%%
\usepackage{eurosym}
\usepackage{vmargin}
\usepackage{amsmath}
\usepackage{graphics}
\usepackage{epsfig}
\usepackage{subfigure}
\usepackage{framed}
\usepackage{multicol}
\usepackage{enumerate}
\usepackage{fancyhdr}

\setcounter{MaxMatrixCols}{10}
%TCIDATA{OutputFilter=LATEX.DLL}
%TCIDATA{Version=5.00.0.2570}
%TCIDATA{<META NAME="SaveForMode"CONTENT="1">}
%TCIDATA{LastRevised=Wednesday, February 23, 201113:24:34}
%TCIDATA{<META NAME="GraphicsSave" CONTENT="32">}
%TCIDATA{Language=American English}

\pagestyle{fancy}
\setmarginsrb{20mm}{0mm}{20mm}{25mm}{12mm}{11mm}{0mm}{11mm}
\lhead{MS4222} \rhead{Kevin O'Brien} \chead{Week 4 Tutorial} %\input{tcilatex}

\begin{document}
\subsection*{Outcomes}
\begin{itemize}
    \item Descriptive Statistics
    \item Boxplots
    \item Binomial Distribution
\end{itemize}
\subsection*{Questions}	
\begin{enumerate}	
\item Compute the range of the following Data Set.

\[\{65,73,89,56,73,52,47\}\]


\item Computing the Sample Mean 
	
	
	\begin{enumerate}[(a)]
		\item Compute the sample mean for the following data set:\[x = \{ 14, 22, 19, 25, 20, 18, 17, 29, 27 \}\]
		
		
		% Mean = 21 
		% n = 9
		% sum = 189
		
		%-----------------------------------%
		
		\item Compute the sample mean for the following data set:
		\[x =\{10, 3, -7, -9, 7, 2, 4, 19, 25, -6\}\]
		
		% sum = 48
		% n = 10
		% mean = 4.8
		
		%-------------------------------------%
		
		
		\item Compute the sample mean for the following data set:
		\[ x =	\{6, 4, 5, 0, -8, 17, -7, 23, 0, 9\}\]
		
		% mean = 4.9
		% n=10
		
	\end{enumerate}

\item 	For each of the following data sets, sort the values into ascending order and state the sample size.
	Then compute the median and interquartile range.
	
	
	\begin{itemize}
		\item[(i)] $\{31, 40, 33, 39, 11, 12, 28, 23, 18, 15, 34, 24\}$ \smallskip
		
		
		% W <- c(31L, 40L, 33L, 39L, 11L, 12L, 28L, 23L, 18L, 15L, 34L, 24L)
		% Answers (n=12)
		% -  Median : 26
		% -  Q1     : 16.5
		% -  Q3     : 33.5
		
		\item[(ii)]	$\{31, 21, 37, 27, 28, 11, 10, 23, 35, 34, 22\}$ \smallskip
		
		
		% X <- c(31L, 21L, 37L, 27L, 28L, 11L, 10L, 23L, 35L, 34L, 22L)
		% 
		% Answers (n=11)
		% -  Median : 37
		% -  Q1     : 21
		% -  Q3     : 34
		
		
		
		\item[(iii)] 	$\{38, 25, 40, 32, 30, 16, 11, 36, 31, 12\}$ \smallskip
		
		% Y <- c(38L, 25L, 40L, 32L, 30L, 16L, 11L, 36L, 31L, 12L)
		% 
		% Answers (n=10)
		% -  Median : 30.5
		% -  Q1     : 16
		% -  Q3     : 36
		
		\item[(iv)] 	$\{28, 13, 33, 23, 25, 26, 40, 16, 38\}$ \smallskip
		
		
		% Z <- c(28L, 13L, 33L, 23L, 25L, 26L, 40L, 16L, 38L)
		% 
		% Answers (n=9)
		% -  Median : 26
		% -  Q1     : 19.5
		% -  Q3     : 35.5
	\end{itemize}
	
	%-------------------------------------%
	
\item Computing Sample Variance and Standard Deviation
	
	\begin{enumerate}[(a)]
		
		\item Compute the sample variance and standard deviation of the following data set:
		\[x =\{5, -11, 9, 20, 18, 21, -8, 19, 8\}\; \; \; ( \bar{x} = 9) \]
		
		\item Compute the sample variance and standard deviation of the following data set:
		\[x =\{32, 20, 15, 49, 30, 28, 33, 35, 34, 44\}
		\; \; \; ( \bar{x} = 32) \]
		
		
		\item Compute the sample variance and standard deviation of the following data set:
		\[
		x =\{19, 16, 35, 34, 27, 17, 20\} \; \; \; ( \bar{x} = 24) \]
		
		
	\end{enumerate}
\newpage	
	\item The following data give the marks of 10 students in a test (out of 20 marks). 
	
	\[12, 17, 7, 11, 18, 6, 14, 15, 11, 9.\]
	
	Calculate the following descriptive statistics

	\begin{itemize}   
		\item[(a)] the median    
		\item[(b)] the mean     
		\item[(c)] the range    
		\item[(d)] the standard deviation 
		\item[(e)] The Inter-Quartile Range
	\end{itemize}

	
\item 
The grades of a mid-term assessment for a class of students is tabulated as follows;
\begin{table}[ht]
\begin{center}
\begin{tabular}{|rrrrrrrrrr|}
\hline
31 & 36 & 37 & 38 & 38 & 38 & 39 & 39 & 39 & 39 \\
41 & 44 & 44 & 45 & 45 & 45 & 46 & 46 & 46 & 47 \\
47 & 47 & 48 & 48 & 51 & 53 & 54 & 55 & 57 & 59 \\
60 & 61 & 61 & 63 & 74 & 75 & 81 & 81 & 82 & 89 \\
\hline
\end{tabular}
\end{center}
\end{table}
\begin{enumerate}[(a)]
% \item Comment on the shape of the histogram. 
%\item Based on the shape of the histogram, what is the best measure of centrality and variability?
\item Construct a box plot for the above data. Clearly demonstrate how all of the necessary values were computed.
\end{enumerate}
\item 
%-----------------------------------------------------------------%
The heights for a group of forty rowing club members are tabulated as follows;
\begin{table}[ht]
\begin{center}
\begin{tabular}{|rrrrrrrrrr|}
\hline
141 & 148 & 149 & 149 & 155 & 156 & 167 & 169 & 169 & 170 \\
171 & 173 & 175 & 176 & 177 & 179 & 182 & 182 & 183 & 183 \\
183 & 184 & 184 & 184 & 185 & 185 & 185 & 186 & 186 & 189 \\
191 & 191 & 191 & 191 & 192 & 192 & 192 & 193 & 194 & 199 \\
\hline
\end{tabular}
\end{center}
\end{table}
\begin{enumerate}[(a)]

\item Construct a box plot for the above data. Clearly demonstrate how all of the necessary values were computed.
\end{enumerate}



\item
I throw a coin 5 times.  Using the Binomial Distribution formula, calculate the probability that
\begin{multicols}{2}
\begin{enumerate}[(a)]
\item  I throw no heads
\item  I throw one head
\item  I throw exactly 3 heads
\item  I throw at least 2 heads
\end{enumerate}
\end{multicols}
\newpage
\item 
A die is thrown 5 times. Using the Binomial Distribution formula, calculate the probability of
\begin{enumerate}[(a)]
\item  Obtaining exactly one six.
\item  Obtaining at least one six.
\item  Obtaining at least one six, given that not more than two sixes are thrown.
\end{enumerate}

\item
The probability that a component produced in a certain factory is defective is 0.02. \item  A batch contains 100
components.
\begin{enumerate}[(a)]
\item  What is the exact distribution of the number of defective components in a batch?
\item  Calculate the probability that none of the components in a batch are defective.
\item  Calculate the probability that there is more than one defective component in a batch.
\end{enumerate}

\item
Suppose X is a binomial random variable with
$X \sim Bin(n, p)$.
\begin{enumerate}[(a)]
\item  Describe, in your own words, what is meant by the expected value of X.
\item  Compute the expected value, the variance and the standard deviation for the following scenarios

\begin{multicols}{2}
\begin{enumerate}[(i)]
	\item $X \sim Bin(n = 10, p = 0.40)$
	\item $X \sim Bin(n = 15, p = 0.25)$
	\item $X \sim Bin(n = 20, p = 0.30)$
	\item $X \sim Bin(n = 50, p = 0.20)$
	\item $X \sim Bin(n = 200, p = 0.10)$
	\item $X \sim Bin(n = 1000, p = 0.01)$
\end{enumerate}
\end{multicols}
\end{enumerate}

	\end{enumerate}
\begin{framed}
\noindent Binomial Distribution Formula:

\[\Pr(X = k) = ^nC_k p^k
(1 − p)^{n−k}\]	
\end{framed}
	\end{document}

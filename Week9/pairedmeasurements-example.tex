\documentclass[a4paper,12pt]{article}
%%%%%%%%%%%%%%%%%%%%%%%%%%%%%%%%%%%%%%%%%%%%%%%%%%%%%%%%%%%%%%%%%%%%%%%%%%%%%%%%%%%%%%%%%%%%%%%%%%%%%%%%%%%%%%%%%%%%%%%%%%%%%%%%%%%%%%%%%%%%%%%%%%%%%%%%%%%%%%%%%%%%%%%%%%%%%%%%%%%%%%%%%%%%%%%%%%%%%%%%%%%%%%%%%%%%%%%%%%%%%%%%%%%%%%%%%%%%%%%%%%%%%%%%%%%%
\usepackage{eurosym}
\usepackage{vmargin}
\usepackage{amsmath}
\usepackage{graphics}
\usepackage{epsfig}
\usepackage{subfigure}
\usepackage{enumerate}
\usepackage{fancyhdr}

\setcounter{MaxMatrixCols}{10}
%TCIDATA{OutputFilter=LATEX.DLL}
%TCIDATA{Version=5.00.0.2570}
%TCIDATA{<META NAME="SaveForMode"CONTENT="1">}
%TCIDATA{LastRevised=Wednesday, February 23, 201113:24:34}
%TCIDATA{<META NAME="GraphicsSave" CONTENT="32">}
%TCIDATA{Language=American English}

\pagestyle{fancy}
\setmarginsrb{20mm}{0mm}{20mm}{25mm}{12mm}{11mm}{0mm}{11mm}
\lhead{MS4222} \rhead{Kevin O'Brien} \chead{Paired Measurements} %\input{tcilatex}

\begin{document}

\section*{Example 1: Paired Difference}
\begin{itemize}
\item An automobile manufacturer collects mileage data for a sample of $n = 10$ cars in various weight categories
using a standard grade of gasoline with and without a particular additive. \item Of course, the engines were tuned to the same
specifications before each run, and the same drivers were used for the two gasoline conditions (with the driver in fact being
unaware of which gasoline was being used on a particular run). \item Given the mileage data on the next slide,  test the hypothesis
that there is no difference between the mean mileage obtained with and without the additive, using the 5 percent level of
significance \item (Remark in lecture: Enough evidence for haulage company to start buying this additive?) \end{itemize}


\begin{center}
\begin{tabular}{|c|c|c|c|c|}\hline
car & with additive & without additive & $d_i$ & $d^2_i$\\\hline
1&36.7&36.2&0.5&0.25\\\hline
2&35.8&35.7&0.1&0.01\\\hline
3&31.9&32.3&-0.4&0.16\\\hline
4&29.3&29.6&-0.3&0.09\\\hline
5&28.4&28.1&0.3&0.09\\\hline
6&25.7&25.8&-0.1&0.01\\\hline
7&24.2&23.9&0.3&0.09\\\hline
8&22.6&22.0&0.6&0.36\\\hline
9&21.9&21.5&0.4&0.16\\\hline
10&20.3&20.0&0.3&0.09\\\hline
\end{tabular}
\end{center}


\begin{itemize}
\item The average of the case wise differences is computed as \[\bar{d} = {\sum d_i \over n}\]
\[ \bar{d} = { 0.5 + 0.1  - 0.4 + \ldots + 0.30 \over 10 }= 0.17 \]
\item Also, using last column, $\sum d^2_i = (0.25 + 0.01 + 0.16 + \ldots + 0.09) = 1.31$
\end{itemize}


\begin{framed}

\noindent \textbf{Sample standard deviation of the case-wise differences}:

\[s_d = \sqrt{ {\sum d_i^2 - n\bar{d}^2 \over n-1}}\]

We know the following:
\begin{itemize}
\item The sample size $n$ which is 10.
\item The average of the case-wise differences. $\bar{d} = 0.17$
\item  $\sum d^2_i = 1.31$
\end{itemize}

\end{framed}


\noindent \textbf{Sample standard deviation  of the case-wise differences}:
\[s_d = \sqrt{ {\sum d_i^2 - n\bar{d}^2 \over n-1}}\]

\[s_d = \sqrt{ { 1.31 - 10(0.17)^2 \over 9}} = 0.337\]

\textbf{The standard error:} \[ S.E.(\bar{d}) = \frac{s_d }{\sqrt{n}} = {0.337 \over 3.16} = 0.107\]



\noindent \textbf{Null and Alternative Hypotheses}:
\begin{itemize}
\item That is, the null hypothesis is:\\
$H_0: \mu_d = 0$ Additive makes no difference to performance\\
$H_1: \mu_d \neq 0$ Additive makes a significant difference to performance \\
\end{itemize}

\noindent \textbf{Test Statistic}:
\begin{itemize}
\item Test Statistic
\[TS =\frac{\bar{d} - \mu_d}{S.E.(\bar{d})} =  \frac{0.17 - 0}{0.107} = 1.59\]
\end{itemize}



\noindent \textbf{Critical value}:
\begin{itemize}
\item $\alpha = 0.05, k = 2$ \item small sample , so $df = n-1 = 9$
\item As with earlier examples in the course, CV is found to be \textbf{2.262} from the statistical tables.
\end{itemize}
\bigskip

\noindent \textbf{Decision Rule}:\\
Is $|TS| > CV$? \\ No, we fail to reject the null hypothesis.
There is no enough evidence to suggest this additive is effective in improving mileages for automobiles.


\end{document}

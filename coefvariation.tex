\subsection{ Coefficient of variation}
What happens if you have two sets of data with two different means and two different standard deviations? How do you decide which set is more spread out? Remember the size of the standard deviation is relative to the mean it is associated with.
The coefficient of variation (cv) is often used to compare the relative dispersion between two or more sets of data. It is formed by dividing the standard deviation by the mean and is usually expressed as a percentage i.e. (multiplied by 100). Again we distinguish between the population and sample coefficient of variation.
Population  :  cv  =   s (100)
m
Sample        : cv  =    s (100)            


The coefficient of variation for different distributions are compared and the distribution with the largest cv value has the greatest spread.
3.2.3 Summary
If the frequency distribution of your data is symmetric, the histogram will be symmetric and the mean and standard deviation should be used to describe the data.
If the frequency distribution of your data is asymmetric (skewed), the histogram will be asymmetric and the median and interquartile range should be used to describe the data.der


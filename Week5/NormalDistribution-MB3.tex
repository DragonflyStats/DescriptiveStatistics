\documentclass[a4paper,12pt]{article}
%%%%%%%%%%%%%%%%%%%%%%%%%%%%%%%%%%%%%%%%%%%%%%%%%%%%%%%%%%%%%%%%%%%%%%%%%%%%%%%%%%%%%%%%%%%%%%%%%%%%%%%%%%%%%%%%%%%%%%%%%%%%%%%%%%%%%%%%%%%%%%%%%%%%%%%%%%%%%%%%%%%%%%%%%%%%%%%%%%%%%%%%%%%%%%%%%%%%%%%%%%%%%%%%%%%%%%%%%%%%%%%%%%%%%%%%%%%%%%%%%%%%%%%%%%%%
\usepackage{eurosym}
\usepackage{vmargin}
\usepackage{amsmath}
\usepackage{graphics}
\usepackage{epsfig}
\usepackage{subfigure}
\usepackage{enumerate}
\usepackage{fancyhdr}

\setcounter{MaxMatrixCols}{10}
%TCIDATA{OutputFilter=LATEX.DLL}
%TCIDATA{Version=5.00.0.2570}
%TCIDATA{<META NAME="SaveForMode"CONTENT="1">}
%TCIDATA{LastRevised=Wednesday, February 23, 201113:24:34}
%TCIDATA{<META NAME="GraphicsSave" CONTENT="32">}
%TCIDATA{Language=American English}

\pagestyle{fancy}
\setmarginsrb{20mm}{0mm}{20mm}{25mm}{12mm}{11mm}{0mm}{11mm}
\lhead{MS4222} \rhead{Kevin O'Brien} \chead{Normal Distribution} %\input{tcilatex}

\begin{document}
%============================================================================================% 
\section*{The Standard Normal (Z) Distribution Tables}
\begin{itemize}
% \item A random variable that has a normal distribution with a mean of zero and a standard deviation of one is said to have a standard normal probability distribution.  It is often nick-named the "z" distribution.


\item Importantly, probabilities relating to the $Z$ distribution are comprehensively tabulated in \textbf{\emph{Murdoch Barnes table 3}}.


\item Given a value of $k$ (with $k$ usually between 0 and 4), the probability of a standard normal "Z" random variable being greater than (or equal to) $k$ is given in Murdoch Barnes table 3 (page 71).

\item Other statistical tables can be used, but they may tabulate probabilities in a different way.
\end{itemize}
%%%%%%%%%%%%%%%%%%%%%%%%%%%%%%%%%%%%%%%%%%%%%%%%%%%%

%---------------------------------------------- %
\subsection*{Using Murdoch Barnes Tables 3}
\begin{itemize}
\item For some value $z_o$, between 0 and 4, the Murdoch Barnes tables set 3 tabulate $P(Z \geq z_o)$
\item Ideally $z_o$ would be specified to 2 decimal places. If it is not, round to the closest value.
\item We call the third digit (i.e. the digit in the second decimal place) the ``second precision".
\end{itemize}


\begin{itemize}
\item To compute the relevant probability we express $z_o$ as the sum of $z_o$ without the second precision (i.e first two digits), and the second precision (the third digit).
\begin{itemize}
\item[$\ast$] For example $1.28 = 1.2 + 0.08$.
\end{itemize}
\item Select the row that corresponds to $z_o$ without the second precision (e.g. 1.2).
\item Select the column that corresponds to the second precision(e.g. 0.08).
\item The value that contained on the intersection is $P(Z \geq z_o)$
\end{itemize}
\subsection*{Example 1}
Find $ P(Z \geq 1.28)$
%------------------------------------------------------------------------%
{
\begin{table}[ht]


%\caption{Standard Normal Distribution } % title of Table
\centering % used for centering table
\begin{tabular}{|c|| c c c c |c| c|} % centered columns (4 columns)
\hline %inserts double horizontal lines
& \ldots & \ldots & 0.006 &0.07&0.08&0.09 \\
%heading
\hline \hline% inserts single horizontal line
\ldots & \ldots & \ldots &\ldots& \ldots &\ldots&\dots \\ % inserting body of the table
1.0 & \ldots & \ldots &0.1446& 0.1423 &0.1401&0.1379 \\ % inserting body of the table
1.1 & \ldots & \ldots&0.1230& 0.1210 &0.1190&0.1170 \\ % inserting body of the table
\hline
1.2 & \ldots & \ldots&0.1038 & 0.1020 &\textbf{0.1003}&0.0985\\
\hline
1.3 & \ldots & \ldots &0.0869& 0.0853 &0.0838&0.0823 \\ % inserting body of the table
\ldots & \ldots &\ldots&\ldots & \ldots &\ldots&\ldots\\
\hline %inserts single line
\end{tabular}
%\label{table:nonlin} % is used to refer this table in the text
\end{table}
}

\noindent By Inspection: $ P(Z \geq 1.28)$ = 0.1003

\subsection*{Example 2}

Find  $P(Z \geq 1.80)$\\

Remark:  1.80 = 1.8 + 0.00
\begin{itemize}
\item The row is 1.8
\item The column is 0.00
\end{itemize}

{
\begin{table}[ht]


%\caption{Standard Normal Distribution } % title of Table
\centering % used for centering table
\begin{tabular}{|c|| c c c c c c|} % centered columns (4 columns)
\hline
& 0.00& 0.01& 0.02& 0.03& ....&  \\ \hline
....&&& && & \\ \hline
1.7& 0.0446& & & ....& &\\ \hline
1.8& \textbf{0.0359}& 0.0351& & ....& & \\ \hline
1.9& 0.0287&&& ....& & \\ \hline
....&&&&& &\\ \hline
\hline %inserts single line
\end{tabular}
%\label{table:nonlin} % is used to refer this table in the text
\end{table}
}

\noindent Answer:  $P(Z \geq 1.80 = 0.0359)$





\subsection*{Example 3}  

Find  $P(Z \geq 1.96)$\\ 

note  1.96 = 1.9 + 0.06


\begin{itemize}
\item The row is 1.9
\item The column is 0.06
\end{itemize}


%------------------------------------------------------------------------%
{
\begin{table}[ht]

%\caption{Standard Normal Distribution } % title of Table
\centering % used for centering table
\begin{tabular}{|c|| c c c c c c|} % centered columns (4 columns)
\hline
&....&0.05&0.06&0.07&....& \\ \hline
....&&&&&  &   \\ \hline
1.8&&&0.0314&&&     \\ \hline
1.9&&0.0256&\textbf{0.025}0&0.0244&&     \\ \hline
2.0&&&....&& &   \\ \hline
....&&&&&&\\ \hline
\end{tabular}
%\label{table:nonlin} % is used to refer this table in the text
\end{table}
}

Answer $P(Z \geq 1.96)  = 0.0250$
%===========================================================%

\newpage
\subsection*{Example 4}
\noindent \textbf{Find $ P(Z \geq 0.60)$}
%------------------------------------------------------------------------%
{
\begin{table}[ht]
%\caption{Standard Normal Distribution } % title of Table
\centering % used for centering table
\begin{tabular}{|c|| c c c c c c|} % centered columns (4 columns)
\hline %inserts double horizontal lines
& 0.00 & 0.01 & 0.02 &0.03&\ldots&\ldots \\
%heading
\hline \hline% inserts single horizontal line \hline
\ldots & \ldots &\ldots &\ldots& \ldots &\ldots&\ldots \\ % inserting body of the table
0.4 & 0.3446 & 0.3409&0.3372 & 0.3336 &\ldots&\ldots\\
0.5 & 0.3085 & 0.3050 &0.3015& 0.2981 &\ldots&\dots \\ % inserting body of the table
0.6 & \textbf{0.2743} & 0.2709&0.2676 & 0.2643 &\ldots&\ldots\\
0.7 & 0.2420 & 0.2389 &0.2358& 0.2327 &\ldots&\dots \\ % inserting body of the table
\ldots & \ldots &\ldots &\ldots& \ldots &\ldots&\ldots \\ % inserting body of the table
\hline %inserts single line
\end{tabular}
%\label{table:nonlin} % is used to refer this table in the text
\end{table}
} \\
\noindent Answer: $ P(Z \geq 0.60) = 0.2743$
\subsection*{Example 5}

\begin{itemize}
\item Find $ P(Z \geq 1.64)$
\item Find $ P(Z \geq 1.65)$
\item Estimate $P( Z \geq 1.645)$
\end{itemize}

\begin{table}[ht]
%\subsection*{Find $ P(Z \geq 1.64)$ and $ P(Z \geq 1.65)$}

%\caption{Standard Normal Distribution } % title of Table
\centering % used for centering table
\begin{tabular}{|c|| c c c c c c|} % centered columns (4 columns)
\hline %inserts double horizontal lines
& \ldots & \ldots & 0.04 & 0.05 &0.06&0.07 \\
%heading
\hline \hline% inserts single horizontal line
\ldots & \ldots &\ldots &\ldots& \ldots &\ldots&\ldots \\ %Checked
1.5 & \ldots & 0.0630&0.0618& 0.0606 &0.0594&\dots \\ % inserting body of the table
1.6 & \ldots &0.0516& \textbf{0.0505} & \textbf{0.0495} &0.0485&\ldots\\
1.7 & \ldots &0.0418 &0.0409& 0.0401 &0.0392&\dots \\ % inserting body of the table
\ldots & \ldots &\ldots &\ldots& \ldots &\ldots&\ldots \\ %Checked
\hline %inserts single line
\end{tabular}
%\label{table:nonlin} % is used to refer this table in the text
\end{table}


\begin{itemize}
\item $ P(Z \geq 1.64) = 0.505$
\item $ P(Z \geq 1.65) = 0.495$ \bigskip
\item $ P(Z \geq 1.645)$ is approximately the average value of $ P(Z \geq 1.64)$ and $ P(Z \geq 1.65)$.
\item $ P(Z \geq 1.645)$ = (0.0495 + 0.0505)/2 = 0.0500. ( i.e. $5\%$ )
\end{itemize}

\newpage

\subsection*{Example 6}

\noindent \textbf{Find $z_A$ such that $ P(Z \geq z_a) = 0.10$}
\begin{itemize}
\item The closest probability value in the tables is $0.1003$.
\item The Z-score that corresponds to $0.1003$ is 1.28.
\item (Row : 1.2 , Column : 0.08)
\item Therefore $z_A  \approx 1.28$
\end{itemize}
\newpage
\begin{table}[ht]
%\caption{Standard Normal Distribution } % title of Table
\centering % used for centering table
\begin{tabular}{|c|| c| c| c| c| c| c|} % centered columns (4 columns)
\hline %inserts double horizontal lines
& \ldots & \ldots & 0.006 &0.07&0.08&0.09 \\
%heading
\hline \hline% inserts single horizontal line
\ldots & \ldots & \ldots &\ldots& \ldots &\ldots&\dots \\ % inserting body of the table
1.0 & \ldots & \ldots &0.1446& 0.1423 &0.1401&0.1379 \\ % inserting body of the table
1.1 & \ldots & \ldots&0.1230& 0.1210 &0.1190&0.1170 \\ % inserting body of the table
1.2 & \ldots & \ldots&0.1038 & 0.1020 &\textbf{0.1003}&0.0985\\
1.3 & \ldots & \ldots &0.0869& 0.0853 &0.0838&0.0823 \\ % inserting body of the table
\ldots & \ldots &\ldots&\ldots & \ldots &\ldots&\ldots\\
\hline %inserts single line
\end{tabular}
%\label{table:nonlin} % is used to refer this table in the text
\end{table}


\end{document}


\documentclass[a4paper,12pt]{article}
%%%%%%%%%%%%%%%%%%%%%%%%%%%%%%%%%%%%%%%%%%%%%%%%%%%%%%%%%%%%%%%%%%%%%%%%%%%%%%%%%%%%%%%%%%%%%%%%%%%%%%%%%%%%%%%%%%%%%%%%%%%%%%%%%%%%%%%%%%%%%%%%%%%%%%%%%%%%%%%%%%%%%%%%%%%%%%%%%%%%%%%%%%%%%%%%%%%%%%%%%%%%%%%%%%%%%%%%%%%%%%%%%%%%%%%%%%%%%%%%%%%%%%%%%%%%
\usepackage{eurosym}
\usepackage{vmargin}
\usepackage{amsmath}
\usepackage{graphics}
\usepackage{epsfig}
\usepackage{framed}
\usepackage{subfigure}
\usepackage{enumerate}
\usepackage{fancyhdr}

\setcounter{MaxMatrixCols}{10}
%TCIDATA{OutputFilter=LATEX.DLL}
%TCIDATA{Version=5.00.0.2570}
%TCIDATA{<META NAME="SaveForMode"CONTENT="1">}
%TCIDATA{LastRevised=Wednesday, February 23, 201113:24:34}
%TCIDATA{<META NAME="GraphicsSave" CONTENT="32">}
%TCIDATA{Language=American English}

\pagestyle{fancy}
\setmarginsrb{20mm}{0mm}{20mm}{25mm}{12mm}{11mm}{0mm}{11mm}
\lhead{Probability Distributions:} \rhead{MathsResource} \chead{Poisson Distribution Example} %\input{tcilatex}

\begin{document}




\section*{Example}
\begin{itemize}
	
	\item Suppose that electricity power failures occur according to a Poisson distribution
	with an average of 2 outages every twenty weeks. \item Calculate the probability that there will
	not be more than one power outage during a particular week.
\end{itemize}

\noindent \textbf{Solution:}

\begin{itemize}
	\item The average number of failures per week is: $m = 2/20 = 0.10$
	
	\item ``Not more than one  power outage" means we need to compute and add the probabilities for ``0 outages" plus ``1 outage".
\end{itemize}



Recall: \[P(X = k) = e^{-m}\times \frac{m^k}{k!}\]


\begin{itemize}
	
	\item $P(X = 0)$
	
	\[P(X = 0) = e^{-0.10} \times \frac{0.10^0}{0!} = e^{-0.10} = 0.9048\]
	
	
	\item $P(X = 1)$
	
	\[P(X = 1) = e^{-0.10}\times \frac{0.10^1}{1!} = e^{-0.10}\times 0.1 = 0.0905\]
	
	\item $P(X \leq 1)$
	
	\[P(X \leq 1) = P(X = 0) + P(X = 1) = 0.9048 + 0.0905 = 0.995\]
	
\end{itemize}

\end{document}

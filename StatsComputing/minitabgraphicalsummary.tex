\documentclass[a4paper,12pt]{article}
%%%%%%%%%%%%%%%%%%%%%%%%%%%%%%%%%%%%%%%%%%%%%%%%%%%%%%%%%%%%%%%%%%%%%%%%%%%%%%%%%%%%%%%%%%%%%%%%%%%%%%%%%%%%%%%%%%%%%%%%%%%%%%%%%%%%%%%%%%%%%%%%%%%%%%%%%%%%%%%%%%%%%%%%%%%%%%%%%%%%%%%%%%%%%%%%%%%%%%%%%%%%%%%%%%%%%%%%%%%%%%%%%%%%%%%%%%%%%%%%%%%%%%%%%%%%
\usepackage{eurosym}
\usepackage{vmargin}
\usepackage{amsmath}
\usepackage{graphics}
\usepackage{epsfig}
\usepackage{subfigure}
\usepackage{framed}
\usepackage{enumerate}
\usepackage{fancyhdr}

\setcounter{MaxMatrixCols}{10}
%TCIDATA{OutputFilter=LATEX.DLL}
%TCIDATA{Version=5.00.0.2570}
%TCIDATA{<META NAME="SaveForMode"CONTENT="1">}
%TCIDATA{LastRevised=Wednesday, February 23, 201113:24:34}
%TCIDATA{<META NAME="GraphicsSave" CONTENT="32">}
%TCIDATA{Language=American English}

\pagestyle{fancy}
\setmarginsrb{20mm}{0mm}{20mm}{25mm}{12mm}{11mm}{0mm}{11mm}
\lhead{MS4222} \rhead{Kevin O'Brien} \chead{Statistical Computing
} %\input{tcilatex}

\begin{document}



\section*{Minitab Summaries}
\begin{itemize}
\item Minitab’s Graphical Summary provides both graphical (a histogram with an overlaid normal curve) and numeric summaries of the raw data and key statistics (mean, median, standard deviation, maximum value, minimum value, and sample size).
\item You can also describe the distribution of the data with graphs, conduct an Anderson-Darling normality test, and obtain confidence intervals for the mean, standard deviation, and median.
\item On the left-hand side, there is a histogram of the data, with a normal curve supervised. Normal distribution is indicated by the curve matching up with the histogram.
\item On the right-hand side, there is a listing of all the main summary statistics (i.e. mean, median etc) for the data set.
	\item Minitab uses the ``Anderson Darling" Test (AD Test) for testing the assumption of normality. It is very similar to the Shapiro-Wilk Test. We interpret the output the exact same way.
	\item The Test Statistic is in the top right corner of the graphical summary.
	
	
\end{itemize}


\begin{figure}[h!]
	\centering
	\includegraphics[width=0.99\linewidth]{images/act1-005}
\end{figure}

\newpage

\subsection*{Minitab Summaries}
\begin{figure}[h!]
	\centering
	\includegraphics[width=0.99\linewidth]{images/graphical-summary4}
	
\end{figure}




\end{document}

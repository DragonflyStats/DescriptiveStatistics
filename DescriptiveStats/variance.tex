\documentclass[a4paper,12pt]{article}
%%%%%%%%%%%%%%%%%%%%%%%%%%%%%%%%%%%%%%%%%%%%%%%%%%%%%%%%%%%%%%%%%%%%%%%%%%%%%%%%%%%%%%%%%%%%%%%%%%%%%%%%%%%%%%%%%%%%%%%%%%%%%%%%%%%%%%%%%%%%%%%%%%%%%%%%%%%%%%%%%%%%%%%%%%%%%%%%%%%%%%%%%%%%%%%%%%%%%%%%%%%%%%%%%%%%%%%%%%%%%%%%%%%%%%%%%%%%%%%%%%%%%%%%%%%%
\usepackage{eurosym}
\usepackage{vmargin}
\usepackage{amsmath}
\usepackage{graphics}
\usepackage{epsfig}
\usepackage{subfigure}
\usepackage{framed}
\usepackage{enumerate}
\usepackage{fancyhdr}

\setcounter{MaxMatrixCols}{10}
%TCIDATA{OutputFilter=LATEX.DLL}
%TCIDATA{Version=5.00.0.2570}
%TCIDATA{<META NAME="SaveForMode"CONTENT="1">}
%TCIDATA{LastRevised=Wednesday, February 23, 201113:24:34}
%TCIDATA{<META NAME="GraphicsSave" CONTENT="32">}
%TCIDATA{Language=American English}

\pagestyle{fancy}
\setmarginsrb{20mm}{0mm}{20mm}{25mm}{12mm}{11mm}{0mm}{11mm}
\lhead{MS4222} \rhead{Kevin O'Brien} \chead{Normal Distribution} %\input{tcilatex}

\begin{document}
	


\section*{Introducing Variance}

Consider the three data sets $X$, $Y$ and $Z$
\begin{itemize}
	\item $X= \{900,925,950,975,1025,1050,1075,1100 \}$
	\item $Y=\{900,905,910,920,1080,1090,1095,1100\}$
	\item $Z=\{900,985,990,995,1005,1010,1015,1100\}$
\end{itemize}

\begin{framed}
\noindent For each of the data sets, the following statements can be verified

\begin{itemize}
	\item The mean of each data set is 1000
	\item There are 8 elements in each data set
	\item The minima and maxima are 900 and 1100 for each set
	\item The range is 200.
\end{itemize}
\end{framed}
\noindent From the plot below, notice how different the three data sets are in terms of dispersion around the mean value.

%%%%%%%%%%%%%%%%%%%%%%%%%%%%%%%%%%%%%%%%%%%%%%%%%%%%%%%%%%%%%%%%%%%%%%%%%%%%%%%%%%%%%%




\begin{center}
	\includegraphics[scale=0.5]{images/2AVariance}
\end{center}

	
	
	\begin{itemize}
		
		\item The (population) variance of a random variable is a non-negative number which gives an idea of how widely spread the values are likely to be; the larger the variance, the more scattered the observations on average.
		
		\item Stating the variance gives an impression of how closely concentrated round the expected value the distribution is; it is a measure of the ``spread" of a distribution about its average value.
		
%	\item For probability distributions, Variance is symbolised by $V(X)$ % or $Var(X)$
		
		\item We distinguish between population variance (denoted $\sigma^2$) and sample variance (denoted $s^2$). For now, we will look only at sample variance.
		
		
	\end{itemize}


%%%%%%%%%%%%%%%%%%%%%%%%%%%%%%%%%%%%%%%%%%%%%%%%%%%%%%%%%%%

\subsection*{The Sample Variance and Sample Standard Deviation}	


\begin{itemize}
	
	\item Sample variance is a measure of the spread of or dispersion within a set of sample data.
	
	\item The sample variance is the sum of the squared deviations from their mean divided by one less than the number of observations in the data set. 
	
	\item For example, for $n$ observations $x1, x2, x3, \ldots , xn$  with sample mean $\bar{x}$, the sample variance is given by 
	
	
	\[ s^2 = { \sum (x-\bar{x})^2  \over n-1}\]
	
	
	\item The sample standard deviation is simply the square root of this value.
	
\end{itemize}




\subsection*{Sample Standard Deviation}
	\begin{itemize}
		\item \textbf{Important:} Standard deviation is the square root of variance
		\item Standard deviation is commonly used in preference to variance because it is denominated in the same units as the mean.
		\item For example, if dealing with time units, we could have a variance of something like $25$ \emph{ square minutes }, whereas the equivalent standard deviation is 5 minutes.
		\item Population standard deviation is denoted  $\sigma$.
 Sample standard deviation is denoted $s$.
	\end{itemize}



% The standard deviation is often preferred to the variance as a descriptive measure because it is in the same units as the raw data e.g. if your data is measured in years, the standard deviation will also be in years whereas the variance will be in years squared.


{
\subsection*{Computing Sample Variance}
	We shall use the following formulae to compute the sample variance of each data set (i.e. $s^2_x$, $s^2_y$ and $s^2_z$) respectively.
	
	\[ s^2_x = { \sum (x-\bar{x})^2  \over n-1}\]
	\[ s^2_y = { \sum (y-\bar{y})^2  \over n-1}\]
	\[ s^2_z = { \sum (z-\bar{z})^2  \over n-1}\]
	\begin{itemize}
		\item Mean for each is 1000: $\bar{x} = \bar{y} = \bar{z}  = 1000$
		\item Sample size of each data set is 8 : $ n=8 $
		\item Therefore $ n-1 = 7$
	\end{itemize}


	\[ s^2_x = {(900-1000)^2 +(925-1000)^2+ \ldots \ldots +(1075-1000)^2+(1100-1000)^2   \over 7}\]
	
	\[ s^2_x = {(-100)^2 +(-75)^2 +(-50)^2+(-25)^2 + (25)^2 +(50)^2 +(75)^2+(100)^2   \over 7}\]
	
	\[ s^2_x = {37500 \over 7}  = 5357.143\]

	\bigskip 
	
\noindent The sample variance of $X$ is $5357.14$ square units. Recall that the sample standard deviation ($s$) is the square root of the variance, so for $X$  the sample standard deviation is $s_x = 73.19$ units.
	
	
	
\noindent Similarly 
	\[ s^2_y = {67050 \over 7}  = 9578.571 \mbox{ square units} \]
	
	\[ s^2_z = {20700  \over 7} = 2957.143 \mbox{ square units} \]
	
\noindent The sample standard deviations for $Y$ and $Z$ are $s_y = 97.87$ units, and $s_z =  54.38$ units respectively.
}

\subsection*{Using \texttt{R}}Using \texttt{R} to compute standard deviation and variance for these data sets.

\begin{framed}
\begin{verbatim}
> X=c(900,925,950,975,1025,1050,1075,1100)
> Y=c(900,905,910,920,1080,1090,1095,1100)
> Z=c(900,985,990,995,1005,1010,1015,1100)
>
> sd(X);sd(Y);sd(Z)
[1] 73.19251
[1] 97.87018
[1] 54.37962
> 
>var(X);var(Y);var(Z)
[1] 5357.143
[1] 9578.571
[1] 2957.143
\end{verbatim}
\end{framed}
\end{document}

\documentclass[a4paper,12pt]{article}
%%%%%%%%%%%%%%%%%%%%%%%%%%%%%%%%%%%%%%%%%%%%%%%%%%%%%%%%%%%%%%%%%%%%%%%%%%%%%%%%%%%%%%%%%%%%%%%%%%%%%%%%%%%%%%%%%%%%%%%%%%%%%%%%%%%%%%%%%%%%%%%%%%%%%%%%%%%%%%%%%%%%%%%%%%%%%%%%%%%%%%%%%%%%%%%%%%%%%%%%%%%%%%%%%%%%%%%%%%%%%%%%%%%%%%%%%%%%%%%%%%%%%%%%%%%%
\usepackage{eurosym}
\usepackage{vmargin}
\usepackage{amsmath}
\usepackage{graphics}
\usepackage{epsfig}
\usepackage{subfigure}
\usepackage{enumerate}
\usepackage{fancyhdr}

\setcounter{MaxMatrixCols}{10}
%TCIDATA{OutputFilter=LATEX.DLL}
%TCIDATA{Version=5.00.0.2570}
%TCIDATA{<META NAME="SaveForMode"CONTENT="1">}
%TCIDATA{LastRevised=Wednesday, February 23, 201113:24:34}
%TCIDATA{<META NAME="GraphicsSave" CONTENT="32">}
%TCIDATA{Language=American English}

\pagestyle{fancy}
\setmarginsrb{20mm}{0mm}{20mm}{25mm}{12mm}{11mm}{0mm}{11mm}
\lhead{MS4222} \rhead{Kevin O'Brien} \chead{Confidence Intervals} %\input{tcilatex}

\begin{document}

%------------------------------------------------------------------------------%

\section*{Confidence Interval for a Mean: Example }

\begin{itemize}
\item For a given week, a random sample of 100 hourly employees selected from a very large number of
employees in a manufacturing firm has a sample mean wage of $\bar{x} = 280$ dollars, with a sample standard deviation of
$s = 40$ dollars.
\item Estimate the mean wage for all hourly employees in the firm with an interval estimate such that we can be 95
percent confident that the interval includes the value of the population mean.
\end{itemize}

\noindent \textbf{Solution}
\begin{itemize}
\item The point estimate in this case is the sample mean $\bar{x} = 280$ dollars.
\item We have a large sample (n=100) and the confidence level is $95\%$. Therefore the quantile  is 1.96.
\item The standard error is computed as follows:

\[ S.E( \bar{x} )  = {s \over \sqrt{n}}  =  {40 \over \sqrt{100}} = 4  \]
\item \textbf{Confidence Interval for mean}

\[
280 \pm (1.96 \times 4)  = (280 \pm 7.84) = (\;272.16\;,\;287.84\;)
\]

\end{itemize}


\end{document}

	\documentclass[a4paper,12pt]{article}
%%%%%%%%%%%%%%%%%%%%%%%%%%%%%%%%%%%%%%%%%%%%%%%%%%%%%%%%%%%%%%%%%%%%%%%%%%%%%%%%%%%%%%%%%%%%%%%%%%%%%%%%%%%%%%%%%%%%%%%%%%%%%%%%%%%%%%%%%%%%%%%%%%%%%%%%%%%%%%%%%%%%%%%%%%%%%%%%%%%%%%%%%%%%%%%%%%%%%%%%%%%%%%%%%%%%%%%%%%%%%%%%%%%%%%%%%%%%%%%%%%%%%%%%%%%%
\usepackage{eurosym}
\usepackage{vmargin}
\usepackage{framed}
\usepackage{amsmath}
\usepackage{graphics}
\usepackage{epsfig}
\usepackage{subfigure}
\usepackage{enumerate}
\usepackage{fancyhdr}

\setcounter{MaxMatrixCols}{10}
%TCIDATA{OutputFilter=LATEX.DLL}
%TCIDATA{Version=5.00.0.2570}
%TCIDATA{<META NAME="SaveForMode"CONTENT="1">}
%TCIDATA{LastRevised=Wednesday, February 23, 201113:24:34}
%TCIDATA{<META NAME="GraphicsSave" CONTENT="32">}
%TCIDATA{Language=American English}

\pagestyle{fancy}
\setmarginsrb{20mm}{0mm}{20mm}{25mm}{12mm}{11mm}{0mm}{11mm}
\lhead{MS4222} \rhead{Kevin O'Brien} \chead{Hypothesis Testing} %\input{tcilatex}

\begin{document}

\section*{Worked Example CI for Difference in Two Means}
A research company is comparing computers from two different companies, X-Cel and Yellow, on the basis of energy consumption per hour. Given the following data, compute a $95\%$ confidence interval for the difference in energy consumption.
\begin{center}
	\begin{tabular}{|c|c|c|c|}
		\hline
		Type & sample size & mean & variance \\ \hline
		X-cel & 17 & 5.353 & 2.743 \\ \hline
		Yellow & 17 & 3.882 & 2.985 \\ \hline
	\end{tabular}
\end{center}
Remark: It is reasonable to believe that the variances of both groups is the same. Be mindful of this.

\subsection*{Solution}
\begin{itemize}
	\item Point estimate : $\bar{x} - \bar{y}$ = 1.469
	\item Standard Error: 0.5805
	\[ S.E.(\bar{x}-\bar{y}) = \sqrt{\frac{2.743}{17} + \frac{2.985}{17}} = \sqrt{0.3369} = 0.5805 \]
	\item Quantile : 1.96 (Large sample, with confidence level of $95\%$.)
\end{itemize}

\[ 1.469  \pm (1.96 \times 0.5805) = (0.3321,2.607) \]


\noindent This analysis provides evidence that the mean consumption level per hour for X-cel is higher than the mean consumption level per hour for Yellow, and that the difference between means in the population is likely to be between 0.332 and 2.607 units.
\newpage
\subsection*{Small Aggregate Sample Size}
From the previous example (comparing X-cel and Yellow) lets compute a 95\% confidence interval when the sample sizes are $n_x=10$ and $n_y=12$ respectively. (Lets assume the other values remain as they are.)
\begin{center}
	\begin{tabular}{|c|c|c|c|}
		\hline
		Type & sample size & mean & variance \\ \hline
		X-cel & 10 & 5.353 & 2.743 \\ \hline
		Yellow & 12 & 3.882 & 2.985 \\ \hline
	\end{tabular}
\end{center}
The point estimate $\bar{x} - \bar{y}$ remains as 1.469. Also we require that both samples have equal variance. As both $X$ and $Y$ have variances at a similar level, we will assume equal variance.
\begin{framed}
%---------------------------------------------------------%
\noindent \textbf{Computing the Confidence Interval}
Standard Error for difference of two means (small aggregate sample)

\[ S.E.(\bar{x}-\bar{y}) = \sqrt{  s^2_p \left({1\over n_x}+{1\over n_y} \right)} \]

Pooled Variance $s^2_p$ is computed as:

\[ s^2_p = \frac{(n_x-1)s^2_x + (n_y-1)s^2_y}{(n_x-1) + (n_y-1)} \]
\end{framed}
%---------------------------------------------------------%



% \subsection{Computing the Confidence Interval}
\begin{itemize} \item Pooled variance $s^2_p$ is computed as:
	
	\[ s^2_p = \frac{(10-1)2.743 + (12-1)2.985}{(10-1) + (12-1)}  = \frac{57.52}{20} = 2.87\]
	
	\item Standard error for difference of two means is therefore
	
	\[ S.E.(\bar{x}-\bar{y}) = \sqrt{  2.87 \left({1\over 10}+{1\over 12} \right)} = 0.726 \]
	
	\item The aggregate sample size is small i.e. 22. The degrees of freedom is $n_x+n_y-2 = 20$.
	From Murdoch Barnes tables 7, the quantile for a $95\%$ confidence interval is 2.086.
	
	\item The confidence interval is therefore
	\[ 1.469  \pm (2.086 \times 0.726) = 1.4699 \pm 1.514 =  (-0.044, 2.984 )  \]
\end{itemize}


%--------------------------------------------------------%
\end{document}


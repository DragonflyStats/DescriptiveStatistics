	\documentclass[a4paper,12pt]{article}
%%%%%%%%%%%%%%%%%%%%%%%%%%%%%%%%%%%%%%%%%%%%%%%%%%%%%%%%%%%%%%%%%%%%%%%%%%%%%%%%%%%%%%%%%%%%%%%%%%%%%%%%%%%%%%%%%%%%%%%%%%%%%%%%%%%%%%%%%%%%%%%%%%%%%%%%%%%%%%%%%%%%%%%%%%%%%%%%%%%%%%%%%%%%%%%%%%%%%%%%%%%%%%%%%%%%%%%%%%%%%%%%%%%%%%%%%%%%%%%%%%%%%%%%%%%%
\usepackage{eurosym}
\usepackage{vmargin}
\usepackage{framed}
\usepackage{amsmath}
\usepackage{graphics}
\usepackage{epsfig}
\usepackage{subfigure}
\usepackage{enumerate}
\usepackage{fancyhdr}

\setcounter{MaxMatrixCols}{10}
%TCIDATA{OutputFilter=LATEX.DLL}
%TCIDATA{Version=5.00.0.2570}
%TCIDATA{<META NAME="SaveForMode"CONTENT="1">}
%TCIDATA{LastRevised=Wednesday, February 23, 201113:24:34}
%TCIDATA{<META NAME="GraphicsSave" CONTENT="32">}
%TCIDATA{Language=American English}

\pagestyle{fancy}
\setmarginsrb{20mm}{0mm}{20mm}{25mm}{12mm}{11mm}{0mm}{11mm}
\lhead{MS4222} \rhead{Kevin O'Brien} \chead{Confidence Intervals} %\input{tcilatex}

\begin{document}
\section*{Difference in Two Samples Means: Worked Examples}
%% Question 5 Part b : Confidence interval for the difference in means of two samples.

The heights of 100 Americans and 50 Spaniards were observed.
The mean and standard deviation of the heights of a sample according to nationality are given below

\begin{center}
\begin{tabular}{|c|c|c|c|} \hline
& Number & Mean & Std. Dev. \\ \hline
American& 100&172&13\\ \hline
Spanish & 50 & 167 & 12\\ \hline
\end{tabular}     
\end{center}


%% - i)        
\noindent Calculate a 95\% confidence interval for the difference between the mean height of all americans and the mean height of all spaniards.


\noindent \textbf{Point Estimate: Observed difference}

\begin{itemize}
	\item Let X denote the heights of americans : $X= 172$
	\item Let Y denote the heights of spaniards : $Y= 167$
	
	\item The difference in the mean of weights : $X-Y= 5$
\end{itemize}


\noindent \textbf{Quantile}

\begin{itemize}
	\item Large sample (both groups are greater than 30).
	
	\item Population variance $\sigma^2$ is unknown. Use $t-$distribution with $\infty$ degrees of freedom.
\end{itemize}


\begin{itemize}
	\item Confidence level is 95\%. Therefore significance levels is 5\%. ( $\alpha =0.05$)
	\item \textit{Remark : Confidence intervals are always two tailed procedures}
	
\item Murdoch Barnes Table 7

\begin{itemize}
	\item[$\ast$] Row: df =  $\infty$
	\item[$\ast$] Column = $\alpha/k$ = 0.05/2 = 0.025
	\item[$\ast$] Quantile =  1.96
\end{itemize}
\end{itemize}



\noindent \textbf{Standard Error}



\[S.E.(X-Y) = \sqrt{\frac{s^2_x}{n_x} + \frac{s^2_y}{n_y}} \]


\[S.E.(X-Y) = \sqrt{\frac{(13)^2}{100} + \frac{(12)^2}{n_y}}  = \sqrt{1.69 + 2.88} = 2.137 \]


\noindent \textbf{Confidence Interval}


\noindent Confidence Interval is therefore

\[95\% CI = 5 \pm (1.96 \times 2.137) =(0.811,9.189)\]


\end{document}


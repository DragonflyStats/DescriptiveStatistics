\documentclass[a4paper,12pt]{article}
%%%%%%%%%%%%%%%%%%%%%%%%%%%%%%%%%%%%%%%%%%%%%%%%%%%%%%%%%%%%%%%%%%%%%%%%%%%%%%%%%%%%%%%%%%%%%%%%%%%%%%%%%%%%%%%%%%%%%%%%%%%%%%%%%%%%%%%%%%%%%%%%%%%%%%%%%%%%%%%%%%%%%%%%%%%%%%%%%%%%%%%%%%%%%%%%%%%%%%%%%%%%%%%%%%%%%%%%%%%%%%%%%%%%%%%%%%%%%%%%%%%%%%%%%%%%
\usepackage{eurosym}
\usepackage{vmargin}
\usepackage{amsmath}
\usepackage{graphics}
\usepackage{epsfig}
\usepackage{subfigure}
\usepackage{framed}
\usepackage{enumerate}
\usepackage{fancyhdr}

\setcounter{MaxMatrixCols}{10}
%TCIDATA{OutputFilter=LATEX.DLL}
%TCIDATA{Version=5.00.0.2570}
%TCIDATA{<META NAME="SaveForMode"CONTENT="1">}
%TCIDATA{LastRevised=Wednesday, February 23, 201113:24:34}
%TCIDATA{<META NAME="GraphicsSave" CONTENT="32">}
%TCIDATA{Language=American English}

\pagestyle{fancy}
\setmarginsrb{20mm}{0mm}{20mm}{25mm}{12mm}{11mm}{0mm}{11mm}
\lhead{MS4222} \rhead{Kevin O'Brien} \chead{Probability} %\input{tcilatex}

\begin{document}
\section*{Probability: Graduating Student Worked Example }
A student can enter a course either as a beginner (73\%) or as a transferring student (27\%).It is found that 62\% of beginners eventually graduate, and that 78\% of transfers eventually graduate. Find: 
\begin{enumerate}[(A)]
	\item the probability that a randomly chosen student is a beginner who will
	eventually graduate
	\item the probability that a randomly chosen student will eventually graduate
\item Compute the probability that a randomly chosen student is either a beginner or will
eventually graduate, or both.
\item If a student eventually graduates, what is the probability that the student entered
as a transferring student?
\item Are the events ``Eventually graduates" and ``Enters as a transferring student"?
statistically independent?
\end{enumerate}
%------------------------------------------------------------------%
\begin{framed}	
\noindent \textbf{Events}
	
	\begin{itemize}
		\item Student enters as a \textbf{\emph{beginner}} (B) : $P(B)=0.73$
		\item Student enters as a \textbf{\emph{transfer}} (T) : $P(B)=0.27$
		\item A student \textbf{\emph{graduates}} (G) : To be determined.
	\end{itemize}
\end{framed}
\textbf{Conditional Probabilities}
\begin{itemize}
	\item Given that student was a beginner, the probability of graduating is $P(G|B)$ = 0.62.
	\item Given that student was a transfer, the probability of graduating is $P(G|T)$ = 0.78.
\end{itemize}

% \frametitle{Review Question 1 :  Probability}
\subsection*{Solution to Part A}
\begin{itemize}
	\item Compute the probability that a randomly chosen student is a beginner who will
	eventually graduate P(B and G)
	\item Recall \textbf{Conditional Probability} formula
	\[P(A|B_ = \frac{P(A \mbox{ and } B)}{P(B)})\]
	\item Re-arranging
	\[P(A \mbox{ and } B) = P(A|B)\times P(B)  \]
	\item Solving
	\[P(B \mbox{ and } G) = P(G|B)\times P(B) = 0.62 \times 0.73  = 0.4526 \]
	\item Similarly, for Transfer students
	\[P(T \mbox{ and } G) = P(G|T)\times P(T) = 0.78 \times 0.27  = 0.2106 \]
\end{itemize} 

\subsection*{Solution to Part B}
\begin{itemize}
	\item Compute the probability that a randomly chosen student will eventually graduate
	\item Graduating class can be subdivided into two groups : graduates who were beginners and grduates who were transfers
	\item Writing this mathematically:
	\[P(G) = P(G \mbox{ and } B) + P(G \mbox{ and } T) \]
	\item Using our previous results
	\[P(G) = 0.4526 + 0.2106 = 0.6632 \]
	\item The probability that a randomly chosen student will eventually graduate is 66.32\%.
\end{itemize} 

\subsection*{Solution to Part C}
\noindent Compute the probability that a randomly chosen student is either a beginner or will
eventually graduate, or both.
\begin{itemize}
	\item From Question 1: P(B) = 0.73, P(G) = 0.6632, P(G and B) = 0.4526
	\item Using the addition rule
	\[P(G \mbox{ or } B) = P(G) + P(B) - P(G \mbox{ and } B)\]
	\item Using our values
	\[P(G \mbox{ or } B) = 0.73 + 0.6632 - 0.4526 = 0.9406\]
\end{itemize}

\subsection*{Solution to Part D}
\noindent If a student eventually graduates, what is the probability that the student entered
as a transferring student?
\begin{itemize}
	\item Writing this mathematically: $P(T|G)$
	\item Using Bayes' formula (in formula sheet)
	\[P(A|B)=  {(P(B|A)\times \frac{P(A)}{ P(B)}} \]
	\item Using Bayes' formula (in formula sheet)
	\[P(T|G)=  {(P(G|T)\times \frac{P(T)}{P(G)}} = \frac{0.78 \times 0.27}{0.6632} = 0.3175 \]
\end{itemize}

\subsection*{Solution to Part E}
\noindent Are the events \textbf{Eventually graduates} and \textbf{Enters as a transferring student}
statistically independent?

\begin{itemize}
	\item If two events A and B are independent, then we can say
	\[P(A \mbox{ and } B) = P(A) \times P(B)\] 
	\item We have the following values:
	\[P(T \mbox{ and } G) = 0.2106\]
	\item Also
	\[P(T) \times P(G) = 0.27 \times 0.6632 = 0.1790\]
	\item Comparing the two values:
	\[P(T \mbox{ and } G) \neq P(T) \times P(G)\] 
	\item We conclude that the events are not independent.
\end{itemize}

			
	
\end{document}

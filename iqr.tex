
\newpage
3.   Percentiles
A percentile is defined as a point below which a certain per cent of the observations lie e.g. the 50th percentile is the point below which half the observations lie. The percentiles that divide the data into four quarters are called: 
Q1       - 25th percentile or lower quartile
Q2       - 50th percentile or median
Q3        - 75th percentile or upper quartile

Interquartile Range (IQR)
The Interquartile Range (Q3 – Q1) is a  measure of variability commonly used for skewed data.
The IQR  the difference between the point below which 25% of your data lie and the point below which 75% of your data lie i.e. Q3  - Q1. 
To find the value of the quartiles, think of Q1 as the middle of the data less than or equal to the median, and of Q3 as the middle of the data greater than or equal to the median.
Use the same technique for calculating the mean to find these values.






Example

Find the three quartiles and the IQR of the following data

 15  34  7  12  18  9  1  42  56  28  13  24  35

•	First sort the data set into ascending order

1 7  9 12 13 15 18  24 28 34 35 42 56

•	Count how many items are in the data set ( answer 13 items)

•	Which value is the second quartile, which is the median (answer: the 7th item, which is 18)

•	Q1  median of data less than or equal to median ( 7 items)

1	7  9 12 13 15 18 
 Answer: the 4th item, which is 12
•	Q3  median of data greater than or equal to median ( also 7 items)

18  24 28 34 35 42 56 
Answer: the 4th item of this 7, which is 34

The three quartiles are therefore Q1 = 12, Q2 = 18, Q3 = 34
The interquartile range is therefore Q3 – Q1 = 22

\newpage
Percentiles
A percentile is defined as a point below which a certain per cent of the observations lie e.g. the 50th percentile is the point below which half (i.e 50%) of the observations lie. 
Quartiles
The percentiles that divide the data into four quarters are called Quartiles: 
Q1       - 25th percentile or lower quartile
Q2       - 50th percentile or median
Q3       - 75th percentile or upper quartile

Interquartile Range (IQR)
The Interquartile Range (Q3 – Q1) is a  measure of variability commonly used for skewed data.
The IQR  the difference between the point below which 25% of your data lie and the point below which 75% of your data lie i.e. Q3  - Q1. 
To find the value of the quartiles, think of Q1 as the middle of the data less than or equal to the median, and of Q3 as the middle of the data greater than or equal to the median.
Use the same technique for calculating the mean to find these values.






Example

Find the three quartiles and the IQR of the following data

 15  34  7  12  18  9  1  42  56  28  13  24  35

•	First sort the data set into ascending order

1 7  9 12 13 15 18  24 28 34 35 42 56

•	Count how many items are in the data set ( answer 13 items)

•	Which value is the second quartile, which is the median (answer: the 7th item, which is 18)

•	Q1  median of data less than or equal to median ( 7 items)

1	7  9 12 13 15 18 
 Answer: the 4th item, which is 12
•	Q3  median of data greater than or equal to median ( also 7 items)

18  24 28 34 35 42 56 
Answer: the 4th item of this 7, which is 34

The three quartiles are therefore Q1 = 12, Q2 = 18, Q3 = 34
The interquartile range is therefore Q3 – Q1 = 22



\subsection*{The Interquartile Range}
\begin{itemize}
\item The interquartile range is the difference between the third quartile and the first quartile.
\item The \textbf{Inter-Quartile Range} is the difference between the First and Third Quartiles.
\[ IQR = Q3-Q1\]
For our example, the IQR is 7 (i.e. 14-7).
\end{itemize}


%=================================================%
%% Measures of Dispersion

The Interquartile Range (IQR)
\[IQR =Q3-Q1\]


\end{document}

	\documentclass[a4paper,12pt]{article}
%%%%%%%%%%%%%%%%%%%%%%%%%%%%%%%%%%%%%%%%%%%%%%%%%%%%%%%%%%%%%%%%%%%%%%%%%%%%%%%%%%%%%%%%%%%%%%%%%%%%%%%%%%%%%%%%%%%%%%%%%%%%%%%%%%%%%%%%%%%%%%%%%%%%%%%%%%%%%%%%%%%%%%%%%%%%%%%%%%%%%%%%%%%%%%%%%%%%%%%%%%%%%%%%%%%%%%%%%%%%%%%%%%%%%%%%%%%%%%%%%%%%%%%%%%%%
\usepackage{eurosym}
\usepackage{vmargin}
\usepackage{amsmath}
\usepackage{graphics}
\usepackage{epsfig}
\usepackage{subfigure}
\usepackage{enumerate}
\usepackage{fancyhdr}

\setcounter{MaxMatrixCols}{10}
%TCIDATA{OutputFilter=LATEX.DLL}
%TCIDATA{Version=5.00.0.2570}
%TCIDATA{<META NAME="SaveForMode"CONTENT="1">}
%TCIDATA{LastRevised=Wednesday, February 23, 201113:24:34}
%TCIDATA{<META NAME="GraphicsSave" CONTENT="32">}
%TCIDATA{Language=American English}

\pagestyle{fancy}
\setmarginsrb{20mm}{0mm}{20mm}{25mm}{12mm}{11mm}{0mm}{11mm}
\lhead{MS4222} \rhead{Kevin O'Brien} \chead{Descriptive Statistics} %\input{tcilatex}

\begin{document}
\section*{The Inter-quartile Range}

\begin{itemize}
    \item The Inter-quartile Range ($Q_3  - Q_1$) is a  measure of variability commonly used for skewed data.
\item The IQR is the difference between the point below which 25\% of your data lie and the point below which 75\% of your data lie i.e. $Q_3  - Q_1$. 
\item To find the value of the quartiles, think of $Q_1$ as the middle of the data less than or equal to the median (lower partition), and of $Q_3$ as the middle of the data greater than or equal to the median (upper partition).
\item Use the same technique for calculating the median to find these values.
\end{itemize}

%=================================================%

\subsection*{Worked Example}



\begin{itemize}
\item Find the three quartiles and the IQR of the following data

\[ X = \{15,  34,  7,  12,  18,  9,  1,  42,  56,  28,  13,  24,  35\}  \]

\item First sort the data set into ascending order

\[X = \{1, 7,  9, 12, 13, 15, 18,  24, 28, 34, 35, 42, 56\}\]


\item Count how many items are in the data set.\\ Answer: 13 items

\item Which value is the second quartile, i.e. the median? \\ Answer: the 7th item, which is 18.
\end{itemize}



\subsection*{The First Quartile}

$Q_1$ is median of data less than or equal to median: 

\[X_{L} = \{1,7,  9, 12, 13, 15, 18\} \]
Answer: the 4th item of 7, which is 12.

\subsection*{The Third Quartile}

$Q_3$ is median of data greater than or equal to median:



\[X_{U} = \{18,  24, 28, 34, 35, 42, 56\}\] 
Answer: the 4th item of 7, which is 34.



\subsection*{The Inter-quartile Range}
The three quartiles are therefore $Q_1 = 12, Q_2 = 18, Q_3 = 34$.
The inter-quartile range is therefore $Q_3 - Q_1 = 22$

%=================================================%
\end{document}
